\documentclass{article}
\usepackage[utf8]{inputenc}
\usepackage{physics}
\usepackage{amsmath,amssymb}
\usepackage{mathtools}
\usepackage{hyperref}
\usepackage[skip=10pt plus1pt, indent=10pt]{parskip}

\DeclareMathOperator{\spn}{span}
\newcommand{\unit}{1\!\!1}

\title{Fusing resource states}
\date{}

\begin{document}
\maketitle
\begin{abstract}
    We introduce the concept of \textit{post selected resource state} (PSRS), a state with a specific number of photons that represents a particular graph state in the logical basis. This class of states models what we can create in the lab with a reasonable optical setups. We show that we can safely fuse two PSRSs to get a new PSRSs. This enables the creation of arbitrarly big tree-like graph states.
\end{abstract}

\tableofcontents

\newpage
\section{Post selected resource states}. 
Let's start by the definition of \textit{post selected resource states} (PSRSs). The main idea is that PSRSs are easily produced in the lab.

Suppose we want to build the graph state $\ket{G}$ made of $Q$ qubits. We will build this with some linear optic setup, using dual rail enconding. So the setup will consist of $2Q$ logical modes corresponding to $\mathcal{H}_Q$ and $A$ ancillary modes corresponding to $\mathcal{H}_A$. In the dual rail enconding we have that to represent one logical qubit $q^{(i)}$ we need two modes $q^{(i)}_Hq^{(i)}_V$:
\begin{align*}
   \ket{F}_{q^{(i)}} \rightarrow \ket{10}_{q^{(i)}_Hq^{(i)}_V} \\
   \ket{T}_{q^{(i)}} \rightarrow \ket{01}_{q^{(i)}_Hq^{(i)}_V}
\end{align*}
We will use the concept of "working in the coincidence basis", that we can formalize introducing the subspace of logical codes:
\begin{align*}
    \mathcal{H}_L = \spn\{\ket{10}_{q^{(1)}_Hq^{(1)}_V}, \ket{01}_{q^{(1)}_Hq^{(1)}_V} \} \otimes \dots \otimes \spn\{\ket{10}_{q^{(Q)}_Hq^{(Q)}_V}, \ket{01}_{q^{(Q)}_Hq^{(Q)}_V} \} \subset \mathcal{H}_Q
\end{align*}
We will also use the projector $\Pi_L$ on this subspace $\mathcal{H}_L$.

Now consider some optical setup that implement some unitary $U$ that creates the graph state $\ket G$ in the coincidence basis. What does this mean more formally? We will consider setups where the input state is:
\begin{align*}
   \ket{in} = \ket{10}_{q^{(1)}_Hq^{(1)}_V} \otimes \dots \otimes \ket{10}_{q^{(Q)}_Hq^{(Q)}_V} \otimes \ket{0}_{a^{(1)}} \otimes \dots \otimes \ket{0}_{a^{(A)}}
\end{align*}
After injecting this state in the setup we get as output $\ket{out} = U\ket{in}$. We will say that the state $\ket{out}$ \textit{well-represents} the graph state $\ket{G}$ if
\begin{align*}
    \ket{G} = \tr_A\left[\Pi_L \otimes \unit_A \ket{out}\bra{out}\Pi_L \otimes \unit_A\right] = \Pi_L \tr_A\left[\ket{out}\bra{out}\right] \Pi_L
\end{align*}.
Clearly both $\ket{in}$ and $\ket{out}$ have $Q$ photons. We will call the $\ket{out}$ state we have produced in the lab \textit{post selected resource state}. This procedure motivates the following definition.

\textbf{Definition (PSRS).} A state $\ket{\psi}$ with its corresponding graph state $\ket{G}$ is said to be a \textit{post selected resource state} if $\ket{\psi}$ well represents $\ket{G}$ and $\ket{\psi}$ has $Q$ photons.

Notice that the unitary $U$ preserves the number of photons so the $\Pi_L \otimes \unit_A \ket{out}$ will always have $0$ photons in the ancillary modes:
\begin{align*}
    \Pi_L \otimes \unit_A \ket{out}  = \ket{\text{something}}_Q \otimes \ket{0}_{a^{(1)}} \otimes \dots \otimes \ket{0}_{a^{(A)}} 
\end{align*}

\newpage
\section{Fusion gates}
Consider four modes $a_H, a_V, b_H, b_V$ (I will omit the $\dagger$s for semplicity). A \textit{fusion gate} consists of two parts. The first part is the following linear substitution:
\begin{align*}
    a_H &\rightarrow d_H \\
    a_V &\rightarrow \frac{c_H-c_V}{\sqrt{2}} \\
    b_H &\rightarrow \frac{c_H+c_V}{\sqrt{2}} \\
    b_V &\rightarrow d_V
\end{align*}
The second part consists in measuring the modes $(c_H, c_V)$ and postselecting the result $(1, 0)$. By construction, fusion is just a linear map $\Phi: \mathcal{H} \rightarrow \mathcal{H'}$ where $\mathcal{H}$ has $n$ modes and $\mathcal{H'}$ has $n' = n-2$ modes.

Let's check how fusion acts on a Fock state of the form $a_H^m a_V^{m'} b_H^{n'}b_V^{n}\ket{\mathbf{0}}$ where $\ket{\mathbf{0}}$ is the vacuum. The first observation is that if we are not in the case where $(m' = 1, n' = 0)$ or $(m' = 0, n' = 1)$ the state will be discarded by the measuring process. In fact in a term of the form
\begin{align*}
    \left(\frac{c_H-c_V}{\sqrt{2}}\right)^{m'} \left(\frac{c_H+c_V}{\sqrt{2}}\right)^{n'}
\end{align*}
the only monomial in the expansion that doesn't contain a $c_V$ is the $c_H^{m'+n'}$ term. But if we want to measure just a photon in the $c_H$ mode we want $m'+n'=1$ and so the two only possible cases are $(m' = 1, n' = 0)$ or $(m'=0, n'=1)$. 

So we have discovered a first rule: if we are not in the case where $(m' = 1, n' = 0)$ or $(m'=0, n'=1)$ we have:
\begin{align}
    \label{destroy}
    a_H^m a_V^{m'} b_H^{n'}b_V^{n}\ket{\mathbf{0}} \xrightarrow{\Phi} 0 
\end{align}
where by $0$ I just really mean the real number 0.

Now fix $(m' = 1, n' = 0)$ or $(m'=0, n'=1)$:
\begin{align}
    \label{keep1}
    a_H^m a_V^{1} b_H^{0}b_V^{n}\ket{\mathbf{0}} \xrightarrow{\Phi} d_H^{m} d_V^{n}\ket{\mathbf{0}} \\
    \label{keep2}
    a_H^m a_V^{0} b_H^{1}b_V^{n}\ket{\mathbf{0}} \xrightarrow{\Phi} d_H^{m} d_V^{n}\ket{\mathbf{0}}
\end{align}

Let's summarize what we discovered about fusion in a more readable form. Let's introduce the notation:
\begin{align*}
   \ket{F(alse)}_a = \ket{10}_{a_Ha_V} \\
   \ket{T(rue)}_a = \ket{10}_{a_Ha_V} \\
   \ket{E(mpty)}_a = \ket{00}_{a_Ha_V}\\
   \ket{O(ne)}_a = \ket{11}_{a_Ha_V}
\end{align*}

Consider the space 
\begin{align}
\label{goodset}
\nonumber
\mathcal{S} &= \spn\{a_H^m a_V^{m'} b_H^{n'}b_V^{n}\ket{\mathbf{0}}, m, m', n', n \leq 1\} \\&= \spn\{\ket{F}_a, \ket{T}_a, \ket{E}_a, \ket{O}_a\} \otimes \spn\{\ket{F}_b, \ket{T}_b, \ket{E}_b, \ket{O}_b\} 
\end{align}
Clearly $\dim\mathcal{S} = 16$. To completely characterize how fusion acts on $\mathcal{S}$ we have to give 16 rules for the basis elements and then by linearity we know how it works on the whole space.

$\star$ \textbf{Group 1.}

\begin{align*}
    \ket{T}_a \xleftrightarrow{\Phi} \ket{T}_b &\implies \ket{T}_d\\
    \ket{F}_a \xleftrightarrow{\Phi} \ket{F}_b &\implies \ket{F}_d\\
    \ket{T}_a \xleftrightarrow{\Phi} \ket{F}_b &\implies 0\\
    \ket{F}_a \xleftrightarrow{\Phi} \ket{T}_b &\implies 0
\end{align*}
These rules are what we expect from a working fusion. In fact when we fuse a monomial according to these rules we are just implementing the right fusion gate:
\begin{align}
    \label{goodfusion}
    \Phi \sim \ket{T}_{d}\prescript{}{a}{\bra{T}}\prescript{}{b}{\bra{T}}+\ket{F}_{d}\prescript{}{a}{\bra{F}}\prescript{}{b}{\bra{F}}
\end{align}

$\star$ \textbf{Group 2.}

\begin{align*}
    \ket{O}_a \xleftrightarrow{\Phi} \ket{O}_b &\implies 0\\
    \ket{E}_a \xleftrightarrow{\Phi} \ket{E}_b &\implies 0\\
    \ket{T}_a \xleftrightarrow{\Phi} \ket{E}_b &\implies \ket{E}_d\\
    \ket{E}_a \xleftrightarrow{\Phi} \ket{F}_b &\implies \ket{E}_d\\
    \ket{F}_a \xleftrightarrow{\Phi} \ket{O}_b &\implies \ket{O}_d\\
    \ket{O}_a \xleftrightarrow{\Phi} \ket{T}_b &\implies \ket{O}_d
\end{align*}
These six rules will turn out to be harmless: in this cases the fusion either destroys the monomial (first two cases) or it doesn't generate a logical acceptable qubit and so the monomial will be discarded by the projection on the logical basis.

$\star$ \textbf{Group 3.}

\begin{align}
    \label{oef}
    \ket{O}_a \xleftrightarrow{\Phi} \ket{E}_b \implies \ket{F}_d\\
    \label{eot}
    \ket{E}_a \xleftrightarrow{\Phi} \ket{O}_b \implies \ket{T}_d
\end{align}
These last two rules are the ones that will give us an hard time in the later discussion, because we get a logical acceptable qubit from two qubits that are not logical.


So we have completely characterized how fusion acts on $\mathcal{S}$: we have given the rules for how fusion acts on its basis elements and by linearity we know how fusion acts on the whole subspace.

Now let's make a crucial observation on what happens when we apply fusion to an element outside $\mathcal{S}$ (so when there is more than one photons in at least one of the $a_H, a_V, b_H, b_V$ or equivalently at least one of $m, m', n', n$ is strictly greater than $1$). This observation will help greatly simplify our reasoning later.

\textbf{Crucial observation (Fusion outside $\mathcal{S}$).} Suppose we have modes\\$q^{(1)}_H, q^{(1)}_V, \dots q^{(Q)}_H, q^{(Q)}_V$ and we start with a state with numbers of photons in each mode given by the array $[n^{(1)}_H, n^{(1)}_V, \dots n^{(Q)}_H, n^{(Q)}_V]$. Suppose that in some mode $H/V$ we have more than one photon: $n^{(\bar{i})}_{H/V} =\bar{g}> 1$. Now suppose that we start fusing qubits: we apply a succession of fusion gates $\{\Phi_{i_1, j_1}, \dots, \Phi{i_N, j_N}\}$, where $\Phi_{i, j}$ naturally means to apply a fusion gate to modes $q^{(i)}_H, q^{(i)}_V,  q^{(j)}_H, q^{(j)}_V$. The crucial observation is that after this $N$ fusion we still get a state of $2(Q-N)$ modes, where there is at least one mode with population $\bar{g}$ or alternatively we destroyed the initial state (we got $0$ at some point).

\textbf{Proof.} The reasoning is the following: suppose that after some fusions we fuse the incriminated qubit $\bar{i}$. We have four possibilities depending on the position of the mode with $\bar{g}$ photons in the fusion. If $m' = \bar{g}$ or $n' = \bar{g}$ the fusion destroys the state because $\bar{g} > 1$ (see the rule in (\ref{destroy})). If $m = \bar{g}$ or $n = \bar{g}$ we either destroy the state (depending on the values of $m'$ and $n'$) or we get a new mode $d_{H/V}$ with $\bar{g}$ photons (see the rules in (\ref{keep1}) and in (\ref{keep2})). The observation can then be proved on induction on the number of fusions $N$.\hfill $\square$

\textbf{Another crucial observation (Number of photons after fusion).} Applying fusion to a state on $2Q+A$ modes with $Q$ photons either destroys the state or we get a state on $2(Q-1)+A$ modes with $Q-1$ qubits.

\newpage
\section{Fusing PSRSs}
Let's now move to the main theorems we wish to prove.

\textbf{Theorem (Fusing two PSRS).} Consider a PSRS $\ket{\psi} \in \mathcal{H}_{Q} \otimes \mathcal{H}_{A}$ that $\textit{well-represents}$ the graph state $\ket{G}$ and another PSRS $\ket{\psi'} \in \mathcal{H}_{Q'} \otimes \mathcal{H}_{A'}$ that $\textit{well-represents}$ the graph state $\ket{G'}$. If we fuse two qubits $q_i, q'_j$ (belonging to $\ket{\psi}, \ket{\psi'}$ respectively) the new state after the fusion $\Phi(\ket{\psi}\otimes\ket{\psi'})$ is itself a PSRS that well represents the graph state obtained fusing the vertices $i$ and $j$ of the graphs $G$ and $G'$.

\textbf{Proof.} The state $\ket{\psi}$ will be the superposition of monomials of the form $[q^{(1)}_H, q^{(1)}_V, \dots, q^{(Q)}_H, q^{(Q)}_V, a^{(1)}, \dots, a^{(A)}]$ (we are just giving the occupation numbers). The state $\ket{\psi'}$ will be the superposition of monomials of the form \\ $[q'^{(1)}_H, q'^{(1)}_V, \dots, q'^{(Q')}_H, q'^{(Q')}_V, a'^{(1)}, \dots, a'^{(A')}]$. So the product $\ket{\psi} \otimes \ket{\psi'}$ will be a superposition of monomials of the form:
\begin{align}
\label{monomial}
\nonumber
&[q^{(1)}_H, q^{(1)}_V, \dots, q^{(Q)}_H, q^{(Q)}_V, a^{(1)}, \dots, a^{(A)}] \otimes [q'^{(1)}_H, q'^{(1)}_V, \dots, q'^{(Q')}_H, q'^{(Q')}_V, a'^{(1)}, \dots, a'^{(A')}] \\ \cong
&[q^{(1)}_H, q^{(1)}_V, \dots, q^{(Q)}_H, q^{(Q)}_V, a^{(1)}, \dots, a^{(A)}, q'^{(1)}_H, q'^{(1)}_V, \dots, q'^{(Q')}_H, q'^{(Q')}_V, a'^{(1)}, \dots, a'^{(A')}]
\end{align}
We have already discussed how fusion acts on monomials and thus by linearity we know how it will act on $\ket{\psi} \otimes \ket{\psi'}$.

The main idea is the following: monomials before fusion of the  form $(\ref{monomial})$ can be logical or not (of course we mean that every qubit $q \in Q$ and $q' \in Q$ is in a logical format). We will call that subspace of logical monomials before fusion $\mathcal{L} = \spn\{\text{logical monomials}\} \subset \mathcal{H}_{Q} \otimes\mathcal{H}_{Q'} \otimes\mathcal{H}_A \otimes\mathcal{H}_{A'}$. Monomials that are logical before fusion transforms well under fusion (in the sense that fusion acts as (\ref{goodfusion})). One can prove that if fusion works well the action on the corresponding graph states $\ket{G}$, $\ket{G'}$ is that of fusing the according vertices (we will not give a proof of this fact here). In other words, if we had $\ket{\psi} \otimes \ket{\psi'} \in \mathcal{L}$ the theorem would be proven.

We will prove that if a monomial is non logical before fusion (is in $\mathcal{L}^{\perp}$), it will be either destroyed by the fusion or after fusion it will still be non logical and so it will be destroyed by the projection.

First observe that if there's an occupation number strictly bigger than $1$, the monomial will be destroyed when we will project on the logical basis of $\mathcal{H}_{Q} \otimes\mathcal{H}_{Q'} \otimes\mathcal{H}_A \otimes\mathcal{H}_{A'}$. This follows from the crucial observation we stated before: also if we have $q^{(i)}_H > 1$  or $q^{(i)}_V > 1$ the monomial gets either destroyed by the fusion or by the projection in the logical basis, because fusion cannot help reducing the number of photons $n>1$ in a mode. So we can just consider monomials with all the occupation numbers less than or equal to $1$.



Thanks to these two observations we just have to study fusion on the monomials with all the occupation numbers smaller or equal to $1$. We perfectly know the rules of fusion on this set here. Let's call $d=(d_H, d_V)$ the new qubit we get after fusion.

Consider a non logical monomial with all the occupation numbers smaller than or equal to one. We have three possible cases based on the $\star$ \textbf{Group 1, 2, 3} classification introduce before:

\textit{The monomial is in $\star$ Group 1.} This means that the qubits $q_i$ and $q_j'$ are in a logical format. So there must exist a qubit untouched by fusion in $Q \backslash \{q_i\} \cup Q' \backslash \{q_j'\}$ that is non logical: the monomial will be destroyed by projection.

\textit{The monomial is in $\star$ Group 2.} In this case we get a new qubit $d$ that is non logical: the monomial will be destroyed by projection.

And now the more tricky part.

\textit{The monomial is in $\star$ Group 3.}  Suppose  for instance we are fusing a monomial obeying to rule $(\ref{eot})$, so a monomial of this kind:
\begin{align*}
   [\dots, q^{(i)}_H=0, q^{(i)}_V=0, q'^{(j)}_H=1, q'^{(j)}_V=1, \dots] = \dots \otimes \ket{E}_{q_i} \otimes \ket{O}_{q'_j}\otimes\dots
\end{align*} 
After fusion we get
\begin{align*}
   [\dots, d_H=0, d_H = 1, \dots] = \dots \otimes \ket{T}_d \otimes \dots
\end{align*} 
Now notice that if the qubit $q_j'$ is in the state $\ket{O}_{q_j'}$ there must exist another qubit $q_k'$ in the state $\ket{E}_{q_k'}$ because $\ket{\psi'}$ is a PSRS and so it must contain $Q'$ photons. To summarize the monomial transforms as:
\begin{align*}
    \dots \otimes \ket{E}_{q_i} \otimes \ket{O}_{q'_j}\otimes\dots \otimes \ket{E}_{q'_k} \otimes \dots
    \xrightarrow{\Phi}\dots \otimes \ket{T}_{d}\otimes\dots \otimes \ket{E}_{q'_k} \otimes \dots
\end{align*}
So at the end this monomial gets destroyed by projection on the logical basis because of the qubit $q_k'$.
The same exact reasoning applies if the monomial obeys to rule $(\ref{oef})$.

We thus conclude that the new state after fusion well represents the new graph with $Q+Q'-1$ vertices obtained by the fusion of graphs $G$ and $G'$. In addition, thanks to the observation on how fusion changes the number of photons in a state, we also conlcude that the new state after fusion will have $Q+Q'-1$ photons and thus it is a PSRS.\hfill$\square$

\textbf{Corollary (Fusing multiple PSRS).} In the lab we will start with a set of $N$ PSRSs of different kinds $\{\ket{\psi_1}, \dots, \ket{\psi_N}\}$. After $N$ fusion we will get a final PSRS. Thanks to the previous theorem we see that is safe to fuse the PSRSs in the way we prefer: fusing two PSRSs always gives us a PSRS. Both the following example procedures works, that of course will lead to different final graph states:
\begin{enumerate}
    \item We can first fuse $\ket{\psi_1}$ with $\ket{\psi_2}$. Then fuse $\ket{\psi_3}$ with this first PSRS obtained by the fusion of the first two and so on.
    \item We can first fuse the first $N/2$ PSRSs in some way and the remaining PSRSs in other way and than fuse the two PSRSs states together.
\end{enumerate}

\textbf{Corollary (Building a tree-like graph state with post selected two vertices PSRS).} Suppose that in the lab we are able to generate $N$ two vertices PSRS. We can fuse them together to get an arbitrary tree-like graph state in the logical basis with $N-1$ vertices.

\textbf{What can't we do?} Suppose we want to generate in the lab a graph state that is not a tree. So given for example a tree-like PSRS $\ket{\psi}$ we try to fuse two qubits $q_i, q_j$ to create a cycle. In this case can get terribly wrong because the monomials in \textbf{$\star$ Group 3} won't be destroyed by the projection after fusion because there is no qubit $q_k$ left in the non logical state $\ket{E}_{q_k}$.

\section{Type II fusion}
Let's introduce a new type of fusion, called \textit{Type II fusion}. Also this kind of fusion is made of two parts. Firstly a linear transformation
\begin{align*}
   a_H &\rightarrow \frac{1}{2}\left[c_H-c_V+d_H+d_V\right] \\
   a_V &\rightarrow \frac{1}{2}\left[-c_H+c_V+d_H+d_V\right] \\
   b_H &\rightarrow \frac{1}{2}\left[c_H+c_V+d_H-d_V\right] \\
   b_V &\rightarrow \frac{1}{2}\left[c_H-c_V-d_H+d_V\right] \\
\end{align*}
and then postselecting states with $(c_H, c_V, d_H, d_V) = (1, 0, 1, 0)$.

Now let's look at how a monomial
\begin{align*}
    a_H^m a_V^n b_H^r b_V^s
\end{align*}
get's transformed. It's clear that if the degree of the monomial is not two it gets destroyed. So let's check the behaviour of the monomials of degree two:
\begin{align*}
    a_H^{2} &\rightarrow 1\\
    a_Ha_V  &\rightarrow 0\\
    a_Hb_H  &\rightarrow 1\\
    a_Hb_V  &\rightarrow 0\\
    a_V^{2} &\rightarrow -1\\
    a_Vb_H  &\rightarrow 0\\
    a_Vb_V  &\rightarrow 1\\
    b_H^{2} &\rightarrow 1\\
    b_H b_V &\rightarrow 0\\
    b_V^{2} &\rightarrow -1
\end{align*}

\section{On the number of outer loops}
Suppose we want to build a tree with $N$ vertices, fusing together $N-1$ Bell pairs. We know discuss some results about the minimal number of outer loops we need.

\textbf{}


\end{document}